\documentclass[../main.tex]{subfiles}
    
\begin{document}

\begin{abstract}
The issue with detecting characters on an image is still a problem today. By
being able to detect handwritten characters, a great amount of documentation
work done by humans  could be replaced by computers. By having this technology
present it could be applied in many fields, for example digitalization of old
documents written by hand. It can also be combined with text-to-speech to help
visually impaired persons to read a larger amount of texts without need of
tactile translation. 

The goal of this project is to to develop a system that is  able to analyse an
image and extract any available text. It should also be able to save the text
in a file and convert the text to speech, with an accuracy of at least 75\%.
The user is going to be able to capture a picture with a graphical interface
which shows live camera feed. Once the picture is captured it is analysed and
printed, it should also be converted to speech.

Existing libraries in Python will be used for image processing and character
recognition. The result was a functioning system with a \textit{TBD}\%
accuracy. It supports all the functions that were meant to be implemented.
Although due to communication errors between teacher and project leader our
scope were reduced and the accuracy of the character recognition was lowered
to 60\%.
\end{abstract}

\end{document}